\documentclass[a4paper,10pt]{article}
\usepackage[english]{babel}
\usepackage[utf8]{inputenc}
\usepackage[toc,page]{appendix}
\usepackage{graphicx}

%Includes "References" in the table of contents
\usepackage[nottoc]{tocbibind}
\usepackage{titling}
\usepackage{setspace}

\parskip .8ex

\setlength{\droptitle}{-15em}

%Begining of the document
\begin{document}

\title{CSCM38: Adv Topic - Artificial Intelligence and Cyber Security - Coursework 1}
\author{Andy Gray\\445348}
\date{10/11/2020}

\maketitle

\section{Introduction}
\label{sec:intro}
	We will be looking at some issues surrounding an advanced topic within natural language processing (NLP).  NLP is a form of artificial intelligence that aims to as the automatic manipulation of natural language, like speech and text, by software \cite{nlp_definition}. However, human language is highly ambiguous. It is also ever-changing and evolving. People are great at producing language and understanding language and are capable of expressing, perceiving, and interpreting very elaborate and nuanced meanings. At the same time, while we humans are great users of language, we are also very poor at formally understanding and describing the rules that govern language \cite{goldberg2017neural}. So if the human language is difficult for humans, the process, therefore, can not be straight forward for computers either. However, some advancements of the years like [give examples of NLP processces].
	
	[Explain/ overview of how NN have become good at NLP]
	
	[Project plan] With the introduction of Artificial Neural Networks and the growth with popularity with frameworks like TensorFlow and PyTorch, we will be looking at how using a deep learning method of NLP might perform compared to a more traditional non-NN version.
	
	[overview of what to expect in assignment]
	First, we will look into the related work, covering what NLP is and some of the more traditional approaches, then we will look at the more recent advancements of NLP and look into how they work. Additionally, we will be looking at what advantages they are claiming to have over the traditional methods and each other.

\section{Related Work}
\label{sec:related_work}

\subsection{Traditional ML Approaches}

The most popular supervised NLP machine learning algorithms are \cite{ml_nlp}:
Support Vector Machines
Bayesian Networks
Maximum Entropy
Conditional Random Field

\subsection{Recurrent NN (RNN) for NLP}
\subsection{BERT for NLP}
\subsection{LSTM for NLP}



\section{Project Plan}
\label{sec:project_plan}


\medskip
\newpage
\begin{appendices}
	
	
\end{appendices}

\newpage

%Sets the bibliography style to UNSRT and imports the 
%bibliography file "samples.bib".
\bibliographystyle{acm}
\bibliography{samples}

\end{document}